\chapter{Backlog (78 US)}
Droits : contributeurs > watchers > utilisateurs enregistr\'es > visiteur => un watcher \`a au moins les m\^emes droits qu'un visiteur.\\
\begin{supertabular}{|p{4.5cm}|p{6.5cm}|p{5cm}|}

\hline
En tant que ... & je souhaite ... & afin de ... \\

\hline
\multicolumn{3}{|c|}{{\LARGE acc\`es}} \\
\hline

visiteur & m'enregistrer et me logger & \\
utilisateur enregistr\'e & modifier mes informations et me d\'econnecter & \\
utilisateur enregistr\'e & pouvoir contribuer \`a plusieurs projets  & \\
utilisateur enregistr\'e & demander \`a rejoindre un projet & \^etre accept\'e par un contributeur de devenir moi m\^eme contributeur \\
contributeur & envoyer une invitation \`a un utilisateur enregistr\'e & lui permettre de valider de de devenir contributeur \\
contributeur & modifier n'importe quelle information de mon projet & faire avancer le projet \\
utilisateur enregistr\'e & demander \`a voir un projet & \^etre accept\'e par un contributeur de devenir watcher \\
contributeur & envoyer une invitation \`a un utilisateur enregistr\'e & lui permettre de valider de de devenir watcher \\
watcher & voir n'importe quelle information du projet & \\
visiteur & voir n'importe quelle information d'un projet public & \\

\hline
\multicolumn{3}{|c|}{{\LARGE projet}} \\
\hline

utilisateur enregistr\'e & cr\'eer un projet public ou priv\'e & \\
contributeur & supprimer mon projet & \\
contributeur & modifier le statut d'un contributeur (d\'eveloppeur, product owner ou scrum master) & d\'efinir des r\^oles au sein du projet \\
(contributeur & \'eventuellement pouvoir modifier les droits des watchers & maintenir la confidentialit\'e de certaines parties de mon projet) \\

\hline
\multicolumn{3}{|c|}{{\LARGE organisation - planning}} \\
\hline

contributeur & cr\'eer un planning des sprints & organiser l'avancement du projet \\
(contributeur & \'eventuellement disposer d'outils pour faciliter la cr\'eation d'un planning des sprints &) \\
watcher & afficher le planning des sprints & \\
contributeur & cr\'eer un diagramme de PERT initial & \\
(contributeur & qu'un diagramme de PERT initial soit g\'en\'er\'e automatiquement \`a partir de mon backlog pond\'er\'e &) \\
watcher & afficher le diagramme de PERT initial du projet & \\
contributeur & cr\'eer et mettre manuellement \`a jour n'importe quand un diagramme de PERT r\'eel du projet (avancement dans la pratique) & \\
(contributeur & qu'un diagramme de PERT r\'eel soit g\'en\'er\'e automatiquement \`a partir du PERT initial et de l'avancement r\'eel du projet & ) \\
(contributeur & afficher les diff\'erences entre les diagrammes de PERT initial et r\'eel & me r\'eajuster l'organisation du projet) \\
watcher & afficher le diagramme de PERT r\'eel du projet & \\
contributeur & cr\'eer un diagramme de GANTT initial & \\
(contributeur & avoir un outil qui me propose plusieurs organisations possibles pour mon GANTT initial & ) \\
watcher & afficher le diagramme de GANTT initil du projet & \\
contributeur & cr\'eer et mettre manuellement \`a jour n'importe quand un diagramme de GANTT r\'eel & \\
(contributeur & avoir un outil qui me propose plusieurs r\'e-organisations possibles pour mon GANTT r\'eel & ) \\
watcher & afficher le diagramme de GANTT r\'eel du projet & \\
contributeur & cr\'eer un planning th\'eorique des versions (dates, contenu) & \\
watcher & afficher le planning th\'eorique des versions & \\
contributeur & cr\'eer et mettre manuellement \`a jour n'importe quand un planning r\'eel des versions & \\
(contributeur & que le planning r\'eel des versions soit mis \`a jour automatiquement (\`a partir des commits ?) & ) \\
watcher & afficher le planning r\'eel des versions & \\
watcher & voir un burn down chart g\'en\'er\'e automatiquement \`a la fin de chaque sprint & \\

\hline
\multicolumn{3}{|c|}{{\LARGE US}} \\
\hline

contributeur & cr\'eer une US & \\
contributeur & supprimer une US & \\
watcher & afficher le co\^ut total d'une US (MAJ auto) & \\
contributeur & modifier le statut d'une US (MAJ auto via t\^aches ou GIT ??) & \\
watcher & voir le statut d'une US (not validated, validated, ready, doing, dev done, done) & \\
contributeur & modifier la priorit\'e d'une t\^ache & \\
watcher &  filtrer les US par statut & \\
watcher & voir la liste des t\^aches d'une US (approche hi\'erarchique) & \\
watcher & voir les dates de d\'ebut et fin d'une US termit\'ee & \\
watcher & voir la date de fin estim\'ee d'une US non done & \\
watcher & voir la date de d\'ebut estim\'ee d'une US non doing & \\
watcher & que les dates de fin d'US r\'evis\'ees dans la page \og{}US\fg{} soient automatiquement mises \`a jour en fonction de l'avanc\'ee du PERT r\'eel & \\

\hline
\multicolumn{3}{|c|}{{\LARGE t\^aches}} \\
\hline

watcher & avoir acc\`es \`a une page \og{}t\^aches\fg{} dans laquelle je trouverais la liste des t\^aches et leurs co\^uts & \\
watcher & voir la liste des t\^aches (approche hi\'erarchique ou ensemble) & \\
contributeur & cr\'eer une nouvelle t\^ache & \\
contributeur & supprimer une t\^ache & \\
contributeur & ajouter des t\^aches aux US (une t\^ache peut appartenir \`a plusieurs US) & \\
contributeur & modifier le co\^ut d'une t\^ache & \\
contributeur & voir la suite de fibonacci pendant le calcul du co\^ut d'une t\^ache & \\
watcher & que chaque modification de co\^ut d'une t\^ache se r\'epercute automatiquement sur toutes les US qui en d\'ependent & \\
contributeur & attribuer un contributeur \`a une t\^ache \`a tout moment & \\
contributeur & modifier le statut d'une t\^ache (MAJ auto via GIT ??) & \\
watcher & voir le statut d'un t\^ache (not ready, ready, doing, dev done ou done) ; & afin de conna\^itre son \'etat d'avancement \\
watcher & filtrer les t\^aches par statut & \\
watcher & filtrer les t\^aches par d\'eveloppeur & \\
watcher & voir les dates de fin de t\^aches initiales et r\'evis\'ees dans la page \og{}taches\fg{} & \\
watcher & que les dates de fin de t\^aches r\'evis\'ees dans la page \og{}taches\fg{} soient automatiquement mises \`a jour en fonction de l'avanc\'ee du PERT r\'eel & \\
watcher & acc\'eder facilement \`a la liste des US d\'ependant d'une t\^ache & \\
watcher & voir le(s) test(s) en rapport avec une t\^ache (lien vers la page \og{}test\fg{} & \\

\hline
\multicolumn{3}{|c|}{{\LARGE tests}} \\
\hline

contributeur & pouvoir modifier la page de tests \`a tout moment & \\
watcher & avoir acc\`es \`a une page \og{}tests\fg{} regroupant les tests par cat\'egorie (unitaires, int\'egration et validation) & \\
watcher & pour chaque test, voir facilement le dev, la date de cr\'eation, la date de derni\`ere ex\'ecution, le nom de celui qui a ex\'ecut\'e la derni\`ere fois et le r\'esultat de la derni\`ere ex\'ecution & \\
watcher & pour chaque ex\'ecution de chaque test, pouvoir acc\'eder \`a la date, le nom du contributeur qui a ex\'ecut\'e, au r\'esultat & \\

\hline
\multicolumn{3}{|c|}{{\LARGE versions}} \\
\hline

contributeur & avoir acc\`es \`a une page \og{}versions\fg{} que je pourrais modifier \`a tout moment & d\'eposer une nouvelle version (d\'enomination, date, contenu [US termin\'ees], \'eventuels bugs / probl\`emes connus \\
contributeur & pouvoir modifier \`a tout moment la liste des \'eventuels bugs / probl\`emes connus & la mettre \`a jour \\
watcher & avoir acc\`es \`a une page \og{}versions\fg{} & conna\^itre toutes les informations sur les version \\

\hline
\multicolumn{3}{|c|}{{\LARGE d\'eveloppeur}} \\
\hline

contributeur & avoir une page \`a mon nom regroupant mon emploi du temps (th\'eorique et r\'eel), mes tests (\'ecrits, d\'evelopp\'es et ex\'ecut\'es) et mes t\^aches (futures, en cours et termin\'ees)  & \\
watcher & voir la page d'un contributeur & conna\^itre son travail dans le projet \\

\hline
\multicolumn{3}{|c|}{{\LARGE accessibilit\'e du code}} \\
\hline

contributeur & entrer l'URL du git du projet & cr\'eer un lien entre l'environnement et le code \\
watcher & pouvoir t\'el\'echarger le code par US, t\^ache(s), test(s), version(s), d\'eveloppeur(s) et projet complet lorsque je suis sur les pages correspondantes & \\

\hline
\multicolumn{3}{|c|}{{\LARGE outils}} \\
\hline

watcher & afficher un wiki contenant contenant la liste de tous les environnements et outils utilis\'es & conna\^itre leurs description g\'en\'erale, description dans le projet, lien, tutos d'installation et de configuration \\
contributeur & modifier le wiki des outils du projet & les mettre \`a jour \\

\hline
\end{supertabular}