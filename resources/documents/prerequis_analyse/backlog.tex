\chapter{Backlog}
%à é è " " ê â î ô à
Droits : contributeurs > watchers > utilisateurs enregistr\'es > visiteur => un watcher \`a au moins les m\^emes droits qu'un visiteur.\\
\begin{supertabular}{|p{4.5cm}|p{6.5cm}|p{5cm}|}

\hline
En tant que ... & je souhaite ... & afin de ... \\

\hline
\multicolumn{3}{|c|}{{\LARGE acc\`es}} \\
\hline

visiteur & m'enregistrer et me logger & \\
utilisateur enregistr\'e & modifier mes informations et me d\'econnecter & \\
utilisateur enregistr\'e & pouvoir contribuer \`a plusieurs projets  & \\
utilisateur enregistr\'e & demander \`a rejoindre un projet & \^etre accept\'e par un contributeur de devenir moi m\^eme contributeur \\
contributeur & envoyer une invitation \`a un utilisateur enregistr\'e & lui permettre de valider de de devenir contributeur \\
contributeur & modifier n'importe quelle information de mon projet & faire avancer le projet \\
utilisateur enregistr\'e & demander \`a voir un projet & \^etre accept\'e par un contributeur de devenir watcher \\
contributeur & envoyer une invitation \`a un utilisateur enregistr\'e & lui permettre de valider de de devenir watcher \\
watcher & voir n'importe quelle information du projet & \\
visiteur & voir n'importe quelle information d'un projet public & \\

\hline
\multicolumn{3}{|c|}{{\LARGE projet}} \\
\hline

utilisateur enregistr\'e & cr\'eer un projet public ou priv\'e & \\
contributeur & supprimer mon projet & \\
contributeur & modifier le statut d'un contributeur (d\'eveloppeur, product owner ou scrum master) & d\'efinir des r\^oles au sein du projet \\
(contributeur & \'eventuellement pouvoir modifier les droits des watchers & maintenir la confidentialit\'e de certaines parties de mon projet) \\

\hline
\multicolumn{3}{|c|}{{\LARGE organisation - planning}} \\
\hline

contributeur & cr\'eer un planning des sprints & organiser l'avancement du projet \\
(contributeur & \'eventuellement disposer d'outils pour faciliter la cr\'eation d'un planning des sprints &) \\
watcher & afficher le planning des sprints & \\
contributeur & cr\'eer un diagramme de PERT initial & \\
(contributeur & qu'un diagramme de PERT initial soit g\'en\'er\'e automatiquement \`a partir de mon backlog pond\'er\'e &) \\
watcher & afficher le diagramme de PERT initial du projet & \\
contributeur & cr\'eer et mettre manuellement \`a jour n'importe quand un diagramme de PERT r\'eel du projet (avancement dans la pratique) & \\
(contributeur & qu'un diagramme de PERT r\'eel soit g\'en\'er\'e automatiquement \`a partir du PERT initial et de l'avancement r\'eel du projet & ) \\
(contributeur & afficher les diff\'erences entre les diagrammes de PERT initial et r\'eel & me r\'eajuster l'organisation du projet) \\
watcher & afficher le diagramme de PERT r\'eel du projet & \\
contributeur & cr\'eer un diagramme de GANTT initial & \\
(contributeur & avoir un outil qui me propose plusieurs organisations possibles pour mon GANTT initial & ) \\
watcher & afficher le diagramme de GANTT initil du projet & \\
contributeur & cr\'eer et mettre manuellement \`a jour n'importe quand un diagramme de GANTT r\'eel & \\
(contributeur & avoir un outil qui me propose plusieurs r\'e-organisations possibles pour mon GANTT r\'eel & ) \\
watcher & afficher le diagramme de GANTT r\'eel du projet & \\
contributeur & cr\'eer un planning th\'eorique des versions (dates, contenu) & \\
watcher & afficher le planning th\'eorique des versions & \\
contributeur & cr\'eer et mettre manuellement \`a jour n'importe quand un planning r\'eel des versions & \\
(contributeur & que le planning r\'eel des versions soit mis \`a jour automatiquement (\`a partir des commits ?) & ) \\
watcher & afficher le planning r\'eel des versions & \\

\hline
\multicolumn{3}{|c|}{{\LARGE US}} \\
\hline

%watcher & voir les dates de fin d'US initiales et révisées dans la page \og{}US\fg{} & \\
%watcher & que les dates de fin d'US révisées dans la page \og{}US\fg{} soient automatiquement mises à jour en fonction de l'avancée du PERT réel & \\


% création d'user stories ;
% achage des coûts totaux des tâches (voir : 2.1.5) de chaque US (MAJ
%auto) ;
% création d'un statut pour chaque US :
% not validated (pas encore une exigence) ;
% validated (exigence, pas encore prête à être executée) ;
% ready ;
% doing ;
% dev done (tests non terminés) ;
% done.
% ltrage des US sur statut (par exemple, ne voir que les exigences ou que
%les US terminées) ;
% possibilité d'acher facilement la liste des tâches de chaque US (approche
%hiérarchique ou ensemble) ;
% possibilité d'acher facilement les dates de début et n de chaque US ;
% achage de la date estimée de n d'US (voir : 2.1.3) ;
% génération automatique d'un burn down chart à la n de chaque sprint
%(voir : 2.1.3).

\hline
\multicolumn{3}{|c|}{{\LARGE t\^aches}} \\
\hline

%watcher & voir les dates de fin de tâches initiales et révisées dans la page \og{}taches\fg{} & \\
%watcher & que les dates de fin de tâches révisées dans la page \og{}taches\fg{} soient automatiquement mises à jour en fonction de l'avancée du PERT réel & \\

% création de tâches ;
% ajout des tâches aux US (une tâche peut appartenir à plusieurs US) ;
% gestion des coûts (suite de bonacci visible pendant le calcul des coûts) ;
% à chaque modication de coût, MAJ automatique du coût de toutes les
%US dont dépend la tâche ;
% possibilité pour tous les contributeurs d'attribuer un développeur à une
%tâche ;
% création d'un statut pour chaque tâche :
% not ready (pas encore prête à être executée) ;
% ready ;
% doing ;
% dev done (tâches codées mais non testées) ;
% done.
% ltrage des tâches sur statut ;
% ltrage des tâches sur dev ;
% achage de la date estimée de n de tâche (voir : 2.1.3) ;
% possibilité d'acher facilement la liste des US dont dépend chaque tâches ;
% possibilité d'accéder au(x) test(s) d'un tâche (lien vers la page de test :
%voir 2.1.6).

\hline
\multicolumn{3}{|c|}{{\LARGE tests}} \\
\hline

%Création d'une page  tests  sur laquelle on trouvera 3 catégories :
% les tests unitaires ;
% les tests d'intégration ;
% les tests de validation.
%Pour chaque test, on pourra visualiser facilement :
% le nom du développeur ;
% la date de création ;
% la date de dernière exécution ;
% le nom de celui qui a exécuté la dernière fois ;
% le résultat de la dernière exécution.
%Il sera également possible de faire apparaître toutes les exécutions avec, pour
%chacune :
% dates ;
% nom du contributeur qui a exécuté ;
% résultat.

\hline
\multicolumn{3}{|c|}{{\LARGE versions}} \\
\hline

%Création d'une page  versions  sur laquelle on trouvera l'historique de
%toutes les versions avec, pour chacune :
% date ;
% contenu (US terminées) ;
% éventuels bugs / problèmes connus (modiable à tout moment par les
%contributeurs).

\hline
\multicolumn{3}{|c|}{{\LARGE d\'eveloppeur}} \\
\hline

%Création d'une page  développeur  sur laquelle on retrouvera :
% les tâches à eectuer / en cours / eectuées ;
% les tests écrits / développés ;
% les tests executés ;
% l'emploi du temps du développeur au cours du projet (théorique et réel).
%

\hline
\multicolumn{3}{|c|}{{\LARGE accessibilit\'e du code}} \\
\hline

%Il sera possible de rentrer l'URL du git du projet. À partir de là, à chaque
%référence au code, on pourra télécharger le / les chier(s) correspondant(s). On
%retrouvera donc le code dans les pages :
% US (toute une US) ;
% tâche (les chiers d'une tâche) ;
% test (les chiers d'un test) ;
% version (une version complète) ;
% développeur (dans la liste des tâches d'un développeur, accès au code de
%chacune des tâches) ;

\hline
\multicolumn{3}{|c|}{{\LARGE outils}} \\
\hline

watcher & afficher un wiki contenant contenant la liste de tous les environnements et outils utilisés & connaître leurs description générale, description dans le projet, lien, tutos d'installation et de configuration \\
contributeur & modifier le wiki des outils du projet & les mettre à jour \\






%%%%%%%%%%%% non fonctionnels : pas vraiment des US

%2.2 Spécications non-fonctionnelles
% légèreté ;
% accessibilité online ;
% ergonomie ;
% garder une trace de chaque changement (état avant modif, date modif,
%état après modif) :
% modications statuts (US, tâches, ...) ;
% changement de dev pour une tâche ;
% état des tests ;
% ...











\hline
\end{supertabular}