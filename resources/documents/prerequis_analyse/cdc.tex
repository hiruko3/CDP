\chapter{Cahier des charges}
%%%%%%%%%%%%%%%%% organisation des r\'epertoires ???????,
\section{Sp\'ecifications fonctionnelles}

\subsection{Acc\`es}
\begin{itemize}
\item gestion inscription / login / logout / configuration de compte (options, suppression, ...) ;
\item seuls les membres inscrits peuvent \^etre ajout\'es comme contributeurs / watchers (voir : \ref{contributeur});
\item seuls les contributeurs d'un projet peuvent modifier le projet ;
\item seuls les watchers peuvent voir un projet priv\'e (voir : \ref{projet_prive}) ;
\item tout le monde (y compris les visiteurs non enregistr\'es) peut voir les projets publics.
\end{itemize}

\subsection{Projets}
\begin{itemize}
\item cr\'eation / suppression de projets (publics / priv\'es)\label{projet_prive} ;
\item ajout de contributeurs \`a un projet\label{contributeur} ;
\item avoir un statut (modifiable) pour chaque contributeur :
\begin{itemize}
\item d\'eveloppeur (statut par d\'efaut) ;
\item scrum master (1 par projet) ;
\item product owner (1 par projet).
\end{itemize}
\item ajout de \og{}watchers\fg{} au projet ;
\item \'eventuellement, g\'erer des acc\`es pour limiter la visibilit\'e des \og{}watchers\fg{} \textcolor{red}{(statuts ?)}.
\end{itemize}

\subsection{Organisation - planning}
\begin{itemize}
\item cr\'eation et affichage d'un planning des sprints\label{sprint} \textcolor{red}{(id\'ees d'outils pour aider \`a la cr\'eation ?)} ;
\item diagramme de PERT initial (possible de le g\'en\'erer automatiquement avec le backlog pond\'er\'e ?) ;
\item (diagramme de PERT mis \`a jour automatiquement en fonction de l'avancement du projet) ;
\item mise \`a jour automatique des dates estim\'ees de fin de t\^aches / US en fonction de l'avanc\'ee du PERT\label{maj_dates_estimees} ;
\item diagramme de GANTT initial (possible d'avoir un outil pour faciliter la cr\'eation ?) ;
\item diagramme de GANTT mis \`a jour en fonction de l'organisation r\'eelle du projet (MAJ manuelle) ;
\item cr\'eation d'un planning th\'eorique des versions (date, contenu) ;
\item planning r\'eel des versions (auto-g\'en\'er\'e \`a partir des commits ?).
\end{itemize}

\subsection{US}
\begin{itemize}
\item cr\'eation d'user stories ;
\item affichage des co\^uts totaux des t\^aches (voir : \ref{tache}) de chaque US (MAJ auto) ;
\item cr\'eation d'un statut pour chaque US :
\begin{itemize}
\item not validated (pas encore une exigence) ;
\item validated (exigence, pas encore pr\^ete \`a \^etre execut\'ee) ;
\item ready ;
\item doing ;
\item dev done (tests non termin\'es) ;
\item done.
\end{itemize}
\item filtrage des US sur statut (par exemple, ne voir que les exigences ou que les US termin\'ees) ;
\item possibilit\'e d'afficher facilement la liste des t\^aches de chaque US (approche hi\'erarchique ou ensemble) ;
\item possibilit\'e d'afficher facilement les dates de d\'ebut et fin de chaque US ;
\item affichage de la date estim\'ee de fin d'US (voir : \ref{maj_dates_estimees});
\item g\'en\'eration automatique d'un burn down chart \`a la fin de chaque sprint (voir : \ref{sprint}).
\end{itemize}

\subsection{T\^aches}
\begin{itemize}
\item cr\'eation de t\^aches\label{tache} ;
\item ajout des t\^aches aux US (une t\^ache peut appartenir \`a plusieurs US);
\item gestion des co\^uts (suite de fibonacci visible pendant le calcul des co\^uts) ;
\item \`a chaque modification de co\^ut, MAJ automatique du co\^ut de toutes les US dont d\'epend la t\^ache ;
\item possibilit\'e pour tous les contributeurs d'attribuer un d\'eveloppeur \`a une t\^ache ;
\item  cr\'eation d'un statut pour chaque t\^ache :
\begin{itemize}
\item not ready (pas encore pr\^ete \`a \^etre execut\'ee) ;
\item ready ;
\item doing ;
\item dev done (t\^aches cod\'ees mais non test\'ees) ;
\item done.
\end{itemize}
\item filtrage des t\^aches sur statut ;
\item filtrage des t\^aches sur dev ;
\item affichage de la date estim\'ee de fin de t\^ache (voir : \ref{maj_dates_estimees});
\item possibilit\'e d'afficher facilement la liste des US dont d\'epend chaque t\^aches ;
\item possibilit\'e d'acc\'eder au(x) test(s) d'un t\^ache (lien vers la page de test : voir \ref{test}).
\end{itemize}

\subsection{Tests}
Cr\'eation d'une page \og{}tests\fg{}\label{test} sur laquelle on trouvera 3 cat\'egories :
\begin{itemize}
\item les tests unitaires ;
\item les tests d'int\'egration ;
\item les tests de validation.
\end{itemize}
Pour chaque test, on pourra visualiser facilement :
\begin{itemize}
\item le nom du d\'eveloppeur ;
\item la date de cr\'eation ;
\item la date de derni\`ere ex\'ecution ;
\item le nom de celui qui a ex\'ecut\'e la derni\`ere fois ;
\item le r\'esultat de la derni\`ere ex\'ecution.
\end{itemize}
Il sera \'egalement possible de faire appara\^itre toutes les ex\'ecutions avec, pour chacune :
\begin{itemize}
\item dates ;
\item nom du contributeur qui a ex\'ecut\'e ;
\item r\'esultat.
\end{itemize}

\subsection{Versions}
Cr\'eation d'une page \og{}versions\fg{} sur laquelle on trouvera l'historique de toutes les versions avec, pour chacune :
\begin{itemize}
\item date ;
\item contenu (US termin\'ees) ;
\item \'eventuels bugs / probl\`emes connus (modifiable \`a tout moment par les contributeurs).
\end{itemize}

\subsection{D\'eveloppeur}
Cr\'eation d'une page \og{}d\'eveloppeur\fg{} sur laquelle on retrouvera :
\begin{itemize}
\item les t\^aches \`a effectuer / en cours / effectu\'ees ;
\item les tests \'ecrits / d\'evelopp\'es ;
\item les tests execut\'es ;
\item l'emploi du temps du d\'eveloppeur au cours du projet (th\'eorique et r\'eel).
\end{itemize}

\subsection{Accessibilit\'e du code}
Il sera possible de rentrer l'URL du git du projet. \`A partir de l\`a, \`a chaque r\'ef\'erence au code, on pourra t\'el\'echarger le / les fichier(s) correspondant(s). On retrouvera donc le code dans les pages :
\begin{itemize}
\item US (toute une US) ;
\item t\^ache (les fichiers d'une t\^ache) ;
\item test (les fichiers d'un test) ;
\item version (une version compl\`ete) ;
\item d\'eveloppeur (dans la liste des t\^aches d'un d\'eveloppeur, acc\`es au code de chacune des t\^aches) ;
\end{itemize}

\subsection{Outils}
Cr\'eation d'une page de type wiki contenant la liste de tous les environnements et outils utilis\'es et pour chacun :
\begin{itemize}
\item une description de leur utilisation g\'en\'erale ;
\item une description de leur utilisation pour le projet ;
\item un lien de t\'el\'echargement ;
\item un tuto d'installations ;
\item un tuto de configuration.
\end{itemize}

\section{Sp\'ecifications non-fonctionnelles}
\begin{itemize}
\item l\'eg\`eret\'e ;
\item accessibilit\'e \textit{online} ;
\item ergonomie ;
\item garder une trace de chaque changement (\'etat avant modif, date modif, \'etat apr\`es modif) :
\begin{itemize}
\item modifications statuts (US, t\^aches, ...) ;
\item changement de dev pour une t\^ache ;
\item \'etat des tests ;
\item ...
\end{itemize}
\end{itemize}